\documentclass[12pt]{article}
\usepackage[francais]{babel}
\usepackage[utf8]{inputenc}
\usepackage{tabularx}
\usepackage[procnames]{listings}
\usepackage{color}
\usepackage[T1]{fontenc}
\usepackage{fancyvrb}
\usepackage{xcolor}
\usepackage{enumitem}

\definecolor{keywords}{RGB}{0,0,255}
\definecolor{comments}{RGB}{0,0,113}
\definecolor{red}{RGB}{160,0,0}
\definecolor{green}{RGB}{0,150,0}
 
\lstset{language=Python, 
        numbers=left,
        basicstyle=\ttfamily\small, 
        keywordstyle=\color{keywords},
        commentstyle=\color{comments},
        showstringspaces=false,
        procnamekeys={def,class}
        tabsize=2,
        frame=single,
        breaklines=true,
        framexleftmargin=10mm,
        xleftmargin=12mm
        }
\title{Compte rendu}

\begin{document}
\maketitle
Par Alex Medina, Adrien Gennevois et Yanis Réveiller.

\tableofcontents
\newpage
\section{Exercices}

a) La formules 5.4 est-elle linéaire en $\Delta$ ?\\
retraite prise au $N^{ieme}$ anniversaire
$$qN = \frac{\Delta}{r}((1+r)^{N+1} - (1+r))$$
Afin de testé la linéarité, on multiplie par y d'un coté et par x de l'autre. Si $x=y$, qN sera linéaire en $\Delta$
$$y\times qN = x\times\frac{\Delta}{r}((1+r)^{N+1} - (1+r))$$
$$y = x\times \frac{qN}{qN}$$

b) Tester la linéarité en $r$
$$x\times qN = \frac{\Delta}{y\times r}((1+yr)^{N+1} - (1+yr))$$
prenons $x=2$, $r=8\%$, $\Delta$ $=100$ et $N=10$
$$2\times qN = \frac{\Delta}{2r}((1+2r)^{N+1} - (1+2r))$$
$$2\times 2473.29 \neq 1564.55$$
qN n'est donc pas linéaire en $r$

c) L'épargnante vers $\frac{\Delta}{2}$ tous les 6 mois
\\avec $r=8\%$, $\Delta$ $=1000$ et $N=25$:
$$qN = 72105.94\$$$
avec $r=8\%$, $\Delta$ $=1\frac{1000}{2}$ et $N=25\times 2$:
$$reff=(1+\frac{0.08}{2})^2-1$$
$$reff=0.0816$$
$$qN' = 328091.62\$$$
\textbf{Apres N années, on retire beaucoup plus si on verse deux fois moins mais deux fois plus fréquemment.}

\subsection{Exercice 3}

\qquad Le taux effectif d'une compagnie est de 18\%, quel pourcentage mensuel celle-ci doit-elle utiliser ? Ce taux sera t-il plus grand ou plus petit que $\frac{18}{12}\% = 1,5\%$?
\\

\qquad Le taux effectif mensuel sera supérieur, en effet les taux effectif mensuel sont toujours supérieur à leurs taux effectif.

\qquad Pour calculer le taux effectif, j'ai utilisé cette formule :

$$tauxEffectif=(1+\frac{18\%}{12})^{12}-1$$
$$19,6\%$$

\qquad Soit le taux effectif mensuel : 
$$\frac{19,6\%}{12}=1,63\%$$

\qquad Le taux est donc bien supérieur à $\frac{18}{12}\%$

\subsection{Exercice 4}

\qquad Une étudiante fait un placement sur 45 ans en plaçant 1000\$ avec un taux à 5\%. Nous allons calculer le placement après ces 45 ans avec des intérêts annuels puis mensuels au taux de $\frac{5}{12}\%$

\qquad La formule pour calculer le placement de la $ i^{eme}$ année par rapport au placement nominal ($P_0$) est :
$$P_i = P_0 * (1+taux)^i$$

\qquad Le placement annuel est donc de:
$$P_{45} = 1000 * (1+5\%)^{45}$$
$$8995,01\$ $$

\qquad Le placement mensuel est de: 9997\$
\\

\qquad Si un autre étudiant souhaite avoir la même somme d'argent après seulement 20 ans, pour trouver l'argent qu'il devra investir, j'ai utilisé la fonction suivante:

\begin{lstlisting}
def ex4B(deltaA, dureeA, taux, duree):
    placementA = placement(deltaA, dureeA, taux)
    delta=1
    while(placement(delta, duree, taux) <= placementA):
        delta += 1
    return delta
\end{lstlisting}

\qquad Cette fonction prend en paramètre la somme initial d'une premère personne, la durée de son placement, ainsi que son taux, et un dernier paramètre pour la durée du placement d'une deuxième personne. Après cela on va chercher à calculer le montant du placement de la deuxième personne.

\qquad Cette fonction retourne 3387\$

\subsection{Exercice 5}

$$p10 = 1000(1+r)^{10}$$

a)b) Une somme de 1000\$ est placée pour 10 ans:
\begin{center}
   \begin{tabularx}{13cm}{ | l | c | c | X |}
     \hline
     r & p10 & rendement annuel $\frac{p10-100}{10}$ & nb d'année pour doubler le rendement annuel \\ 
     \hline
     0.06 & 1790.85 & 79.08 & 31ans \\ 
     \hline
     0.08 & 2158.92 & 115.89 & 25ans \\ 
     \hline
     0.1 & 2593.74 & 158.37 & 22ans \\ 
     \hline
   \end{tabularx}
 \end{center}

c) Avec des intérêts simples:
\begin{center}
   \begin{tabularx}{10cm}{ | l | c | c |}
     \hline
     r & p10 intérêt simple & rendement annuel $\frac{p10'-100}{10}$  \\ 
     \hline
     0.06 & 1600 & 60  \\ 
     \hline
     0.08 & 1800 & 80  \\ 
     \hline
     0.1 & 2000 & 100 \\ 
     \hline
   \end{tabularx}
 \end{center}
 
d) Le rendement annuel reste constant chaque année avec les intérêts simples. Il ne peut donc pas doubler.\\

Fonctions utilisées:\\
\begin{lstlisting}
'''
    Calcul du gain total d'un placement  une duree 
    en fonction d'un taux d'interet compose et du montant initial P0
'''
def placement(P0, duree, taux):
    taux = taux /100
    return P0*(1+taux)**duree

'''
    Calcul du gain total d'un placement apres une duree 
    en fonction d'un taux d'interet simple et du montant initial P0
'''
def placementSimple(P0, duree, taux):
    taux = taux /100
    return P0+(P0*taux*duree)   

'''
    Calcule le rendement annuel d'interet compose
    avec delta correspondant au montant du depot annuel
'''
def rendAnnuel(delta, duree, taux):
    pla = placement(delta, duree, taux)
    return (pla-delta)/duree

'''
    Calcule le rendement annuel d'interet simple
    avec delta correspondant au montant du depot annuel
'''    
def rendAnnuelSimple(delta, duree, taux):
    pla = placementSimple(delta, duree, taux)
    return (pla-delta)/duree

'''
    Calcule l'annee pour laquelle le rendement annuel aura double
    Avec interets composes
    et avec delta correspondant au montant du depot annuel
'''
def anneeDoubleRendAnnuel(delta, duree, taux):
    x=duree
    while(rendAnnuel(delta,x,taux) < 2*rendAnnuel(delta, duree, taux)):
        x=x+1
    return x 
\end{lstlisting}

\subsection{Exercice 6 }

On cherche à savoir quand la moitié de l'emprunt sera remboursé, pour cela nous avons créer la fonction suivante : 
\begin{lstlisting}
def demi(N , ra , p0) : 
    r = ra / 12 // on calcul le taux mensuel 
    mens = r * p0 *( (1+r) ** (12*N) ) / (( (1+r) ** (12 * N) ) - 1 )
    for k in range(0,12*N+1):
        pk = p0*(1+r)**k - mens * (((1+r) ** k )- 1)/r
        print (k,round(pk,3))
        if(pk <= (p0/2)):
            return ("la moitie est rembourse au mois" + str(k))
\end{lstlisting}
Pour savoir quand la moitié est remboursé il suffit de calculer la somme des mensualité,
toutefois on oublie pas que le prêt augmente à chaque mois avec le taux.Ainsi on vérifie 
quand cette somme dépasse la moitié de l'emprunt initial.
\newpage
\subsection{Exercice 7}

a) Une étudiante de 20 ans met de coté 1000\$ par an avec un taux annuel de 10\%. A ces 65 ans, elle aura:\\
$$q_{65-20} = q_{45} = \frac{1000((1+0.1)^{46} - (1+0.1))}{0.1} = 790 795.32$$

b) si l'étudiante veut être millionnaire: 
$$\frac{x}{0.1}(1.1^{46}-1.1) > 1000000$$
$$x > \frac{1000000\times 0.1}{1.1^{46}-1.1}$$
$$x > 1264.55\$$$

\subsection{Exercice 8}
Emprunt avec intérêt simple durant 5 ans à 10%:
$$p_5 = p_0(1+0.1\times 5)$$
Emprunt avec intérêt composé durant 5 ans à 7%:
$$p_5'=p_0(1+0.07)^5$$
$p_5'$ est le plus avantageux:
$$p_0(1+0.1\times 5) > p_0(1+0.07)^5$$
$$1.5>1.07^5$$
$$1.5>1.40$$
$$p_5>p_5'$$

\subsection{Exercice 9}

a) Emprunt de 100 000\$, remboursement sur 25 ans, contrat de 3 ans à 12\%.

\begin{itemize}[label=$\bullet$]
\item versement mensuel $\Delta$: $p_0=100000$, $r=0.12$ et $rm=\frac{0.12}{12}=1\%$
$$ \Delta = rm \times p_0 \times \frac{(1+rm)^{12N}}{((1+rm)^{12N}-1)}$$
$$\Delta=1053.22\$$$
\item solde après 3 ans: \\d'après la simulation d'emprunt sur le site, après 36 mois on a
$$resteAPayer = 97707.27\$$$
\end{itemize}
% mettre le code de la simulation et l'expliquer
b) Il reste 97 707.27\$ à payer, sur 22 ans avec un contrat de 8\% durant 5 ans:
\begin{itemize}[label=$\bullet$]
\item versement mensuel $\Delta'$: $p_0'=97707.27$, $r'=0.08$ et $rm'=\frac{0.08}{12}=0.67\%$\\
$$\Delta = 790.19\$$$
\item solde après 5 ans: 
$$resteaPayer = 87736.73\$$$
\end{itemize}

\subsection{Exercice 10}

\qquad Pour deux prêts hypothécaires accordé à deux taux différent pour le même montant et sur la même durée, c'est le plus petit taux qui aura remboursé le plus, en effet celui avec le plus grand taux va rembourser plus d'intérêts au début du remboursement, donc arriver à la moitié de l'hypothèque, il lui restera à payer plus et il aura aussi payé plus lors de la moitié du remboursement.
\newpage
\subsection{Exercice 11}

\qquad D'après les tables d'hypothécaire, le versement mensuel d'un prêt de 40000\$ à 8\% sur 15 ans sera de 379,26\$. Pour un montant de 42000\$ le versement serait entre 379,26\$ et 426,67\$, j'ai pu déduire la somme exacte en utilisant la formule suivante:
$$ \Delta = r_mp_0 \frac{(1+r_m)^{12N}}{(1+r_m)^{12N}-1}$$
$$ soit\ \Delta = 398,22\$ $$

Pour calculer le versement d'un prêt d'un montant de 40000\$ sur 15 ans avec cette fois ci un taux effectif, il faut d'abord calculer le taux effectif soit:
$$ r_{eff} = (1+ \frac{8\%}{12})^{12}-1 $$
$$ 8,3\%$$

J'ai ensuite utilisé la formule précédente avec ce nouveau taux pour trouver $\Delta = 385,965\$ $

\subsection{Exercice 12 }

Dans cette exercice on va chercher à comparer le temps de remboursement avec paiement aux deux semaines avec les autres.
On prend la méthode de calcul de taux du canada pour calculer le taux aux deux semaines, on divise 13 car le remboursement de 13 fois doit être équivalent aux remboursement du taux normal sur la moitié de l'année :
\begin{equation}
r2s = e^{(ln(1+r/2)/13} -1
\end{equation}
\begin{equation}
r2s = e^{(ln(1+0,07 / 2)/13} -1
\end{equation}
\begin{equation}
r2s \simeq 0.13353354314079535
\end{equation}

On calcul maintenant les mensualité : 
\begin{equation}
mens = rs * p0 * ( ( (1+rs) ** (26*N) ) / ( (1+rs) ** (26*N) -1 ) ) 
\end{equation}

\subsection{Exercice 13 }

Pour générer la table d'appendice nous avons écrit un programme avec 2 fonctions :  
\begin{itemize}[label=$\bullet$]
\item appendice : qui calcul le total payé pour 1 case en fonction de la période, du montant et du taux.
\begin{lstlisting}
def apendice(periode, montant, taux):
    rm = exp( log( 1 + (taux/2) ) / 6 ) - 1;
    res = rm * montant * ( ( (1+rm)** (12 * periode) ) / ( (1+rm) ** (12 * periode ) - 1 ))
    return res
\end{lstlisting}
\item table qui génère la table en fonction du taux, du montant maximum et de la durée maximum.
\begin{lstlisting}
def table(taux, montantMax, DureeMax):
    listePeriode = []
    for k in range(1, DureeMax+1):
        listePeriode.append(k)
    
    listeMontant = []
    for l in range(1000, montantMax+1000, 1000 ):
        listeMontant.append(l)
        
    tableApendice = []
    for i in range(int(montantMax/1000)):
        tableApendice.append([0] * DureeMax)
        
    for m in range(int(montantMax/1000)):
        for p in range(DureeMax):
             valeur = apendice(listePeriode[p] , listeMontant[m] , taux)
             tableApendice[m][p] = valeur
             print("montant = " + str(listeMontant[m]) + ", annee = " + str(listePeriode[p]) + ", valeur = "+ str(valeur))
    return tableApendice
\end{lstlisting}
\end{itemize}
\newpage
\section{Simulation}
Nous avons réalisé un site permettant de simuler le remboursement d'un emprunt et une épargne.
\end{document}